\hypertarget{index_Overview}{}\section{Overview}\label{index_Overview}
Rocket\-Robot is a multi-\/featured robot simulator. It simulates robots, targets, and obstacles in an environment, with the goal of robots seeking and finding their targets. When a robot finds its target, both disappear from the simulation. In addtion, stimuli or light sources can be placed in the environment. There are several types of robots. Simple robots have sensors that only seek the target, and ignore obstacles. Complex robots have sensors for lights, robots, obstacles, and the target. Note that obstacle sensors include the wall as an 'obstacle', straight ahead of the sensor. For both robot types, the wheels speeds are controlled directly by the sensor readings. Complex robots can have each type of sensor scaled or connected in any pattern, this is controllable from the 'New complex robot settings' tab in the interface. Neural network robots use a feed-\/forward neural network for control. The network takes the target, robot, and obstacle settings as inputs, and outputs the wheel speeds. The network is loaded from a file, specified in the 'New neural network robot settings' tab in the interface. \hypertarget{index_Installation}{}\subsection{Installation}\label{index_Installation}
You can execute the 'gorobot' executable directly from the src or bin folder, but it is easier to directly install it. This allows you to run it as 'rocketrobot $<$optional filename$>$=\char`\"{}\char`\"{}$>$' from any folder. To install on a Linux computer, simply run the install script from the main folder. This soft-\/links a script to your $\sim$/bin folder. This requires that you have a $\sim$/bin folder on your path, if not you can create one and add it to your .bashrc or .cshrc. To uninstall, you can simply run the uninstall script. \hypertarget{index_Interface}{}\subsection{User Interface}\label{index_Interface}
The Rocket\-Robot user interface is fairly simple. The control panel allows you to start, stop, pause, resume, and quit the simulation. The reset button reverts to the previous time it was started if the simulation was opened from a file, or randomly generates a new simulation if the current simulation is random. The 'Refresh settings' button reloads the configuration files (described later). The open/save tab also lets you open or save the current state. You can also provide a file to open as the command-\/line argument. There are several examples simulations for this in the examples folder.

The user inteface is fairly straightforward for adding and removing objects. There are several control tabs from which the number of obstacles, robots, and lights can be controlled. Objects are added and removed in a F\-I\-L\-O manner from the interface. Targets are always created with a robot, but not all robots need to have a target. Decreasing the number of robots through the control panel deletes the target paired with the removed robot.

For more information about an object (location, speed, orientation, radius), you can click on an object. You can also drag and drop objects to move them, or drag them off the screen to delete them. The radius off the selected object can be changed with the up and down arrow keys, the orientation with the left and right keys, and the speed with the + and -\/ keys. You can also middle-\/click to paste a copy of the selected object. Note that copying a robot creates a new robot with the same target (if any.) You cannot copy targets.\hypertarget{index_Optimization}{}\section{Neural network optimization}\label{index_Optimization}
A major portion of the project is the neural network-\/driven robot, and the associated optimization algorithims. To optimize the neural network robot, a seperate simulation class, \hyperlink{classOptimizeSimulation}{Optimize\-Simulation}, was created. Optimization employs a genetic algorthim, in which a pool of possible networks is maintained, along with their performance. The performances are calculated by performing a large number of runs of (consistant) random simulations, and summing the total time for all robots to find all targets. In addition, several spesfic test cases are run for each network to test some interesting cases (e.\-g. navigating a simple maze, etc.) Pairs of networks are randomly selected from the pool (with one network from a better subset of the pool.) These are combined by taking half the weights from each and merging them into a new network, then making more random tweaks. This is then tested and added to the pool if it is better than the worst performance in the pool. The best performance in the pool is the optimal network.

To run the optimization, run ./optimize from the scripts folder. The temporary and intermediate files get saved to runtime/neuralnetwork. This includes the optimization log, pool performances, optimial network, pool networks, and several others.\hypertarget{index_Implementation}{}\section{Implementation}\label{index_Implementation}
The various objects are all subclasses of \hyperlink{classPhysicalObject}{Physical\-Object}, which handles basic behaviors and attributes. This class is virtual, so that spesific bevhaviors for collisions, etc are handled individually, and can optionally call back to the default handlers in \hyperlink{classPhysicalObject}{Physical\-Object}. Each type of robot is a subclass of \hyperlink{classRobot}{Robot}, which is also virtual, as it has virtual functions to get new wheel speeds given the sensor readings. Everything else, including reading from sensors and collision behavior, is handled directly by the robot class.

There is also an environment namespace with which all objects regester, and are automaticly removed when they are destructed. An iterator over all objects can be requested from the environment, and they can also be accessed by id. The environment also contains functions for detecting collisions between walls and objects. A util namespace provides an easy way of adding and removing objects, managing stacks so that objects are added and removed in F\-I\-L\-O order. It also contains functionality to open and save the current state to and from files.

The \hyperlink{classSimulation}{Simulation} class handles the user interface, providing a wrapper for the functions in util. It also acts as a driver, requesting updates and redraws from all objects. \hypertarget{index_Motion}{}\subsection{Object motion}\label{index_Motion}
Object movement is handled by \hyperlink{classPhysicalObject}{Physical\-Object}. It has an update\-Members function which advances all objects by a distance caluclated from the speed, after calling the virtual update function. The robot class update sets the position and speed from the calculated wheel speeds. The update\-Members function is called for each object from advance in util, which in turn is called from advance in \hyperlink{classSimulation}{Simulation} which is triggered by a glut timer function. \hypertarget{index_Graphics}{}\subsection{Graphics}\label{index_Graphics}
A graphical display of the simulation is rendered by Open\-G\-L. Most of the graphics functionallity is in the artist namespace, which has simple functions for drawing circles and lines, which are used by more complex functions for different kinds of objects. There is a virtual function in \hyperlink{classPhysicalObject}{Physical\-Object} to render each object as a circle, and this is overridden by \hyperlink{classRobot}{Robot} and \hyperlink{classLightSource}{Light\-Source}. These are called by simulation similarly to update\-Position. \hypertarget{index_Configuration}{}\subsection{Configuration system}\label{index_Configuration}
Most settings are loaded through the configuration system, see \hyperlink{configuration_8h}{configuration.\-h} for more details. This allows settings to be stored externally, and avoids needing to recompile often when making small changes. \hyperlink{classConfiguration}{Configuration} values can also be set from the command line.\hypertarget{index_Licence}{}\section{Licence}\label{index_Licence}
This project has been developed by Lucas Kramer, Carl Bahn, Himawan Go, and Xi Zhang. It has been extended by Lucas Kramer to add neural network-\/controlled robots and made the interface more useable. Copyright (C) 2015 Lucas Kramer, Carl Bahn, Himawan Go, and Xi Zhang

This program is free software\-: you can redistribute it and/or modify it under the terms of the G\-N\-U General Public License as published by the Free Software Foundation, either version 3 of the License, or (at your option) any later version.

This program is distributed in the hope that it will be useful, but W\-I\-T\-H\-O\-U\-T A\-N\-Y W\-A\-R\-R\-A\-N\-T\-Y; without even the implied warranty of M\-E\-R\-C\-H\-A\-N\-T\-A\-B\-I\-L\-I\-T\-Y or F\-I\-T\-N\-E\-S\-S F\-O\-R A P\-A\-R\-T\-I\-C\-U\-L\-A\-R P\-U\-R\-P\-O\-S\-E. See the G\-N\-U General Public License for more details.

You should have received a copy of the G\-N\-U General Public License along with this program. If not, see \href{http://www.gnu.org/licenses/}{\tt http\-://www.\-gnu.\-org/licenses/}. 